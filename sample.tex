\documentclass[submit,techrep,noauthor]{ipsj}
\usepackage{luatexja}
\usepackage{luatexja-compat}
\usepackage{luatexja-fontspec}

\begin{document}

\title{プログラマーは疑似コードの夢を見るか}
\etitle{Do Programmers Dream of Pseudocode?}

\affiliate{空高}{架空高等学校\\Imaginary High School}
\affiliate{虚大}{虚構大学\\Fictional University}

\author{仮之太郎}{Karino Tarou}{空高}[karino@example.ed.jp]
\author{嘘野花子}{Usono Hanako}{虚大}[karisome@example.ac.jp]

\begin{abstract}
    現代のプログラマは日々、実際のプログラミング言語と教育用擬似コードの狭間で葛藤している。大学入学共通テスト「情報」で用いられる擬似コード記法は、実用性を度外視した抽象的な表現であり、現実のプログラミング実践とは乖離している。本研究では、プログラマが擬似コードという人工的な言語体系に向き合う際の認知的ジレンマを分析し、真のプログラミング能力と試験対策のための形式的知識との間に存在する本質的な差異について考察する。
\end{abstract}
\begin{eabstract}
    Modern programmers face a daily struggle between real programming languages and educational pseudocode. The pseudocode notation used in the Common Test for University Admissions "Informatics" represents an abstract expression that disregards practicality and diverges from actual programming practice. This study analyzes the cognitive dilemma that programmers encounter when dealing with pseudocode as an artificial linguistic system, and examines the essential differences between genuine programming competence and formal knowledge required for exam preparation.
\end{eabstract}
\maketitle

\section{はじめに}

はじめに光があった。ぴかぁ。

\section{このファイルについて}

タイトルはもちろんアレのパクリである\cite{ブレードランナー}。おもしろいパロディが書ける力があればと思うのだが,私にそれを期待するのは無駄である。

とりあえず関連ファイルをGitHub\cite{レポジトリ}に置いておくので,どんどん良くして言ってもらえると全学会員が喜ぶと考えられる。まだ見ぬあなたにありがとう。

\section{責任逃れ}

私自身はほとんど\texttt{ipsj.cls}や\texttt{ipsjtech.sty}のコードを読まずに,エラーのもぐらたたきをしただけなので,どんな問題が潜んでいるかわからない。締め切りギリギリに「どないしてくれるねん」と言われても困るので,余裕を持って執筆することが望まれる。関係各所のみなさん,ごめんなさい。そうやって自分の首を締めるのはやめなさい,とよく言われる。

\section{おわりに}

あなたはよくこう言っていた。終わりは始まり。

\nocite{*}

\bibliographystyle{ipsjunsrt}
\bibliography{sample}

\end{document}
